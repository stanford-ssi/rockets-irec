% Adapted from ME310 document template
%%%%%%%%%%%%%%%%%%%%%%%%%%%%%%%%%%%

%%%%%%%%%BEGIN DOCUMENT STYLE SETTINGS%%%%%%%%%%%
% Don't modify this stuff unless you know what you're doing...
% We are using the "memoir" class, a widely used set of macros book-like documents.
% If you get errors that you are missing the "memoir" package you can download and 
% install it:   http://www.ctan.org/tex-archive/macros/latex/contrib/memoir/

% memoir document class for standard USA letter paper, printed one side
\documentclass[11pt,letterpaper,oneside]{memoir}
\chapterstyle{section}
\usepackage{graphicx}      			  % standard LaTeX graphics 
\usepackage{color}               	  % support for colored fonts
\usepackage{url}  \urlstyle{same}     % deal with url strings in bibliography
\usepackage{gensymb}
\usepackage{wrapfig}
\usepackage[font={small,it}]{caption}

%More special packages to help deal with long requirements tables 
%that might span multiple pages.
\usepackage{multirow} %deal with merged cells in tables
\usepackage{supertabular}
\usepackage{longtable}
\usepackage{morefloats}

\usepackage[pdftex,           %hyperlink cross references, etc.
    pdfsubject={ME112 Documentation},
    colorlinks={true},
    linkcolor={black},
    citecolor={blue},
    bookmarksopenlevel=1,
]{hyperref}				  

%The file "designreport.sty" should be in the same directory as this file.
% It contains formatting for page setup, titlepage, glossary, references, etc.
\usepackage{designreport}           
%%%%%%%END DOCUMENT STYLE SETTINGS%%%%%%%%%%


%%%%%%%%%%BEGIN TITLE PAGE%%%%%%%%%%%%%%%%
%Replace the strings below with what's right  for you.

%%Insert your Document Title here. Use \\ to force a newline.
\title{Rockets IREC Systems Engineering Handbook}

%% If you don't want it to use the printing date, replace "\today"
%% with the date that you want.
\date{\today}
%%%%%%%%%%%END TITLE PAGE%%%%%%%%%%%%%


%%%%%%%%%BEGIN ANY CUSTOM ABBREVIATIONS%%%%%%
% Define any useful abbreviations to save typing.
% For example:
\def\pmt{ {\em papier m\^{a}ch\`{e}} }  %Define "\pmt" to print "Papier Mache" with accents +1space
%%%%%%%%%END CUSTOM ABBREVIATIONS%%%%%%%%%


%%%%%%%%%%%%%%%%%%%%%%%%%%%%%%%%%%%%%%%%
%   BEGIN THE MAIN DOCUMENT
%%%%%%%%%%%%%%%%%%%%%%%%%%%%%%%%%%%%%%%%
\begin{document}

%If you want a figure on the cover page, this is where it goes.
%9 cm is about max figure height before messing up title spacing.
\begin{figure}[t]
\centering
  %An example cover image
  \includegraphics[height= 9cm]{Figures/logo.png}
%\vspace{3 cm}    %Use this instead when you have no cover picture 
\end{figure}

%%%%%%%%%%%%%%%%%%%%%%%%%%%%%%%%%%%%%%
%Make the title page using arguments defined above.
\titlep

%%%%%%%%%%%%%%%%%%%%%%%%
% Load file "Executive.tex" for the Executive summary.
% Remember, this is a stand-alone section for executives to read.
\include{1Executive}

%%%%%%%%%%%%%%%%%%%%%%%%
% TOC and LOF are automatically generated -- Note that sometimes have to "compile" Latex THREE
% times to update the main .aux files, the TOC etc. files, and finally the PDF output with all changes
% propagated to the printout.
% Make Table of Contents title smaller than a normal Chapter heading:
\renewcommand{\chaptitlefont}{\normalfont\Large\bfseries}
\newpage
\tableofcontents*  %asterisk to prevent it from getting a number
\renewcommand{\chaptitlefont}{\normalfont\Huge\bfseries}
%%%%%%%%%%%%%%%%%%%%%%%%

%Very short Background section for ME112 reports
\chapter{Background}
\label{sec:background}


%%%%%%%%%%%%%%%%%%%%%%%%%%%%%%%%
% Design Development 
\chapter{Design Description}
\label{sec-description}
   %Results of benchmarking, need-finding, CFP, CEP etc.
% Design Development 
\chapter{Analysis of Performance}
\label{sec-analysis-performance} %Analysis of the performance, based on test data
% Design Development 
\chapter{Bird 2.0: Redesign}
\label{sec-redesign}
   %A redesign based on what you now know
% Design Development 
\chapter{Conclusions}
\label{sec-conclusions}

%% Appendices are below

%%%%%%%%%%%%%%%%%%%%%%%%%%%%%%%%
\include{Description}  %Description of the design and some explanation of its evolution


%%%%%BEGIN BIBLIOGRAPHY%%%%%%%%%
%%%% If you don't have references, comment out the \bibliography stuff below
%%
% This is a two-step process in which you first create a ".bib" file, which is processed
% by bibtex into a ".bbl" file for loading into the document. 
% Many online journals and databases now have a feature to automatically download Bib
% citations.  GoogleScholar also produces .bib citations for references.
%
% Alternatively, you can create your own bibliography list by hand.
% In that case, comment out the last line above and replace with  \input{OurBibFile} 
%%%%%END BIBLIOGRAPHY%%%%%%%%%

%%%%%BEGIN APPENDIX SECTIONS%%%%%%%%%%%%%%%
\begin{appendices}
 % Appendices are set up same as chapter sections
\chapter{Appendices}
       % There could be multiple appendix files like this
 \end{appendices}
 %%%%%END APPENDIX SECTIONS%%%%%%%%%%%%%%%
 
\end{document}

